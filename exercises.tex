\documentclass[14pt]{article}

\usepackage{amssymb}
\usepackage{mathtools}
\usepackage{wasysym}

\let\oldemptyset\emptyset{}
\let\emptyset\varnothing{}

\begin{document}

\section*{Chapter 2}

\subsection*{Exercise 2.1.1.2}
$\mathcal{P}(A) = \emptyset, \{1\}, \{2\}, \{3\}, \{1,2\}, \{1,3\}, \{2,3\}, \{1,2,3\}$

\subsection*{Exercise 2.1.1.3}
\paragraph{a.}
A function $PR \to RG$.
\paragraph{b.}
Yes, there are probably many many-to-one connection points.

\subsection*{Exercise 2.1.2.5}
\paragraph{a.}
$2 \mapsto 4$
\paragraph{b.}
$0 \mapsto 0$
\paragraph{c.}
Not applicable; $-2 \not\in \mathbb{N}$.
\paragraph{d.}
$5 \mapsto 25$
\paragraph{e.}
The symbol $\to$ associates the domain and codomain, while $\mapsto$
associates a particular member of the domain to a particular member of
the codomain.

\subsection*{Exercise 2.1.2.6}
$\operatorname{im}(f) = \{y_1, y_2, y_4\}$

\subsection*{Exercise 2.1.2.8}
$f(A) = \{0,1,4,9\}$

\subsection*{Exercise 2.1.2.10}
$f \circ x \colon \{\smiley\} \to Y$ and $\smiley \mapsto f(x)$.

\subsection*{Exercise 2.1.2.12}
\paragraph{a.}
$2^5 = 32$
\paragraph{b.}
$5^2 = 25$

\subsection*{Exercise 2.1.2.13}
\paragraph{a.}
Take $A := \{\smiley\}$.

\end{document}
