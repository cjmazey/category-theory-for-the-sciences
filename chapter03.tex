\section*{Chapter 3}


\exercise{3.1.1.4}
$4 \times 3 = 12$ elements


\exercise{3.1.1.8}
\paragraph{a.}
\begin{align*}
  (a,b,c)\mapsto(a+b,c); (x,y)\mapsto xy
  &= (a,b,c)\mapsto ac+bc \\
  (a,b,c)\mapsto(a\cdot b,a\cdot c); (x,y)\mapsto x+y
  &= (a,b,c)\mapsto ab+ac
\end{align*}
does not commute
\paragraph{b.}
\begin{align*}
  x\mapsto(x,0); (a,b)\mapsto(a\cdot b) &= x\mapsto 0 \\
  id_\mathbb{Z} &= x\mapsto x
\end{align*}
does not commute
\paragraph{c.}
\begin{align*}
  x\mapsto(x,1); (a,b)\mapsto(a\cdot b) &= x\mapsto x \\
  id_\mathbb{Z} &= x\mapsto x
\end{align*}
commutes


\exercise{3.1.1.13}
Here is a relationship between them:
\begin{equation*}
  \Hom{Set}(A,X) \times \Hom{Set}(A,Y) \cong \Hom{Set}(A,X \times Y)\,.
\end{equation*}


\exercise{3.1.1.14}
\paragraph{a.}
\begin{equation*}
  s \colon X \times Y \to Y \times X \coloneqq
  \left\langle \pi_2 , \pi_1 \right\rangle \,.
\end{equation*}
\paragraph{b.}
\begin{thm}
  The swap map $s$ above is an isomorphism.
\end{thm}
\begin{proof}
  Note that
  \begin{align*}
    \left\langle p_2 , p_1 \right\rangle \cdot
    \left\langle \pi_2 , \pi_1 \right\rangle \cdot
    p_2
    &= \left\langle p_2 , p_1 \right\rangle \cdot
      \pi_1 \\
    &= p_2
  \end{align*}
  and similarly with $p_1$ and $\pi_2$.

  The universal property for products with
  $p_1$ and $p_2$ implies that
  \begin{equation*}
    \left\langle p_2 , p_1 \right\rangle \cdot
    \left\langle \pi_2 , \pi_1 \right\rangle =
    \id{Y \times X}
  \end{equation*}
  by unicity.

  The other direction is similar.
\end{proof}


\exercise{3.1.2.4}
Yes, it is conceivable that
$\lceil\mathrm{a\ phone}\rceil =
\lceil\mathrm{a\ cell\ phone}\rceil \sqcup
\lceil\mathrm{a\ landline\ phone}\rceil$.


\exercise{3.1.2.9}
\paragraph{a.}
$f(n) = \lvert n \rvert$.
\paragraph{b.}
$A = \mathbb{N},
X = \{\,n \in \mathbb{Z} \,\vert\, n \geqslant 0\,\},
Y = \{\,n \in \mathbb{Z} \,\vert\, n < 0\,\},
X \sqcup Y \cong \mathbb{Z}$.


\exercise{3.1.2.12}
$\Hom{Set}(X,A) \times \Hom{Set}(Y,A) \cong \Hom{Set}(X \sqcup Y,A)$


\exercise{3.1.2.15}
Let $c,d$ be objects.  Then
\begin{equation*}
  \llangle c \sqcup d \rrangle \coloneqq
  \textrm{``either $\llangle c \rrangle$, labled as such, or $\llangle
    d \rrangle$, labled as such.''}
\end{equation*}

The inclusions $c \rightarrow c \sqcup d \leftarrow d$ may be read
``is, when appropriately labled.''
\newline\newline
\begin{tikzpicture}
  \matrix [types]
  {
    \node (W) {a wav file}; & &
    \node (A) {an audio file}; & &
    \node (M) {an mp3 file}; \\
  };
  \path [aspects]
  (W) edge node [above,text width=2cm] {is, with the\\``.wav''\\extension} (A)
  (M) edge node [above,text width=2cm] {is, with the\\``.mp3''\\extension} (A);
\end{tikzpicture}


%%% Local Variables:
%%% mode: latex
%%% TeX-master: "exercises"
%%% End:
